\documentclass[a4paper, 10pt]{article}
\hyphenpenalty=8000
\textwidth=125mm
\textheight=185mm

\usepackage{graphicx}
\usepackage{alltt}
\usepackage{amsmath}
\usepackage[hidelinks, pdftex]{hyperref}

\pagenumbering{arabic}
\setcounter{page}{1}
\renewcommand{\thefootnote}{\fnsymbol{footnote}}
\newcommand{\doi}[1]{\href{https://doi.org/#1}{\texttt{https://doi.org/#1}}}

\begin{document}

\begin{center}
Nonlinear Analysis: Modelling and Control, Vol. vv, No. nn, YYYY\\
\copyright\ Vilnius University\\[24pt]
\LARGE
\textbf{Operations Vectoriales en la Clase `Escena` para Manipulación y Análisis de Puntos en un Espacio 2D}\footnote{Este trabajo fue apoyado por el grant No.\ xxxx.}\\[6pt]
\small
\textbf{Nelson Martinez, Alexdith Ariza}\\[6pt]
Institución de los autores, dirección\\
\href{mailto:author@somewhere.host}{author@somewhere.host}\\[6pt]
Recibido: fecha\quad/\quad
Revisado: fecha\quad/\quad
Publicado online: fecha
\end{center}

\begin{abstract}
Este trabajo presenta una implementación en Python de una clase `Escena` que modela un espacio 2D con un fondo y un punto. La clase permite realizar varias operaciones vectoriales, como el desplazamiento y la rotación de puntos, así como el cálculo del producto escalar y la norma vectorial. Estas operaciones se ilustran a través de tres ejemplos de escenas, donde se realizan transformaciones sobre los puntos en un entorno visual. Este enfoque permite estudiar y manipular puntos en un espacio 2D de forma sencilla y eficiente.
\vskip 2mm

\textbf{Palabras clave:} operaciones vectoriales, rotación, desplazamiento, norma, producto escalar.
\end{abstract}

\nocite{2009ProcDETAp}

\section{Introducción}\label{s:1}
En este trabajo se presenta una implementación de una clase en Python, denominada `Escena`, que permite modelar un espacio 2D con un fondo y un punto. La clase facilita la realización de diversas operaciones vectoriales como el desplazamiento y la rotación de puntos, así como el cálculo de la norma y el producto escalar. Estas herramientas pueden ser útiles en el contexto de análisis matemático, simulaciones de sistemas dinámicos o incluso en el campo de la visualización científica.

\section{Ejemplos de ecuaciones, algoritmos y figuras}\label{s:2}
A continuación se presenta la implementación de las ecuaciones y algoritmos que permiten realizar las operaciones vectoriales sobre el punto de la escena:

\subsection{Expresiones matemáticas complejas}\label{s:2.1}
Como ejemplo de cómo se presentan las operaciones matemáticas en este trabajo, consideremos la rotación y el desplazamiento de un punto en el espacio 2D:

\begin{equation}\label{eq:1}
   \dot{X}=F(X),\qquad \dot{Y}=G(Y,X),
\end{equation}
donde \(X \equiv \{x_1, x_2, \dots, x_d\}\) es el vector de estado del sistema que actúa como el "driver", y \(Y \equiv \{y_1, y_2, \dots, y_r\}\) es el vector de estado del sistema respondedor.

\subsection{Operaciones vectoriales sobre puntos}\label{s:2.2}
El siguiente fragmento muestra cómo se realizan las operaciones de desplazamiento y rotación sobre el punto en la clase `Escena`:

{\small
\begin{alltt}
# Desplazar el punto
escena1.desplazar_punto(np.array([2, 3]))

# Rotar el punto 20 grados
escena2.rotar_punto(20)
\end{alltt}}

\subsection{Ejemplo visual de una figura}\label{s:2.3}
El siguiente gráfico ilustra el resultado de aplicar las transformaciones a los puntos en las diferentes escenas. Se muestra cómo un punto se mueve a través de un fondo de diferentes colores, y cómo las operaciones de rotación y desplazamiento afectan a su posición.

\begin{figure}[ht]
\centering
\includegraphics[width=5.8cm]{1_2.pdf}\quad
\includegraphics[width=5.8cm]{1_1.pdf}
\caption{Resultados de las transformaciones sobre los puntos en las escenas.}\label{fig:1}
\end{figure}

\paragraph{Contribuciones de los autores.} Todos los autores (Nelson Martinez y Alexdith Ariza) han contribuido de la siguiente manera: metodología, Nelson Martinez; análisis formal, Nelson Martinez y Alexdith Ariza; software, Alexdith Ariza; validación, Alexdith Ariza; redacción – preparación del borrador original, Nelson Martinez y Alexdith Ariza; redacción – revisión y edición, Nelson Martinez y Alexdith Ariza. Todos los autores han leído y aprobado la versión final del manuscrito.

\paragraph{Conflictos de interés.} Los autores declaran que no existen conflictos de interés.

\paragraph{Agradecimientos.} Queremos agradecer al Dr. N. Sergejeva por los fragmentos del artículo \cite{2014SergejevaN}.

\bibliographystyle{NAplain}
\bibliography{sample}

\end{document}
\documentclass{article}
\usepackage{graphicx} % Required for inserting images

\title{simulacion}
\author{alexdith ariza }
\date{March 2025}

\begin{document}

\maketitle

\section{Introduction}

\end{document}
