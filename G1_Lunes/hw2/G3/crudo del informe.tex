\documentclass{article}
\usepackage{amsmath, amssymb, graphicx}
\usepackage{xcolor, listings}
\usepackage{hyperref}

% Configuración para código fuente
\lstdefinestyle{mystyle}{
    backgroundcolor=\color{lightgray},
    commentstyle=\color{green},
    keywordstyle=\color{blue},
    numberstyle=\tiny\color{gray},
    stringstyle=\color{red},
    basicstyle=\ttfamily\footnotesize,
    breakatwhitespace=false,
    breaklines=true,
    captionpos=b,
    keepspaces=true,
    numbers=left,
    numbersep=5pt,
    showspaces=false,
    showstringspaces=false,
    showtabs=false,
    tabsize=2
}
\lstset{style=mystyle}

\title{Transformaciones en una Escena 2D}
\author{[carlos jeronimo, robbyel elias, andrus lopez]}
\date{\today}

\begin{document}

\maketitle

\section{Introducción}
Este documento presenta una implementación en Python para manejar una escena 2D que incluye un círculo y puntos simulando ojos. Se incluyen funciones para escalar, desplazar y rotar la escena.

\section{Código Fuente}
A continuación, se muestra el código en Python:

\begin{lstlisting}[language=Python, caption=Código Python para manejar la escena]
import numpy as np
import matplotlib.pyplot as plt
import matplotlib.patches as patches

class Escena:
    def _init_(self, radio=5, color='darkblue'):
        """
        Inicializa la escena con un círculo centrado en el origen y dos puntos simulando ojos.
        """
        self.radio = radio
        self.color = color
        self.puntos = np.array([[-1.5, 2], [1.5, 2]])  # Posiciones de los ojos

    def dibujar_escena(self):
        """
        Dibuja la escena con el círculo, los ejes y los puntos.
        """
        fig, ax = plt.subplots()
        ax.set_xlim(-self.radio - 1, self.radio + 1)
        ax.set_ylim(-self.radio - 1, self.radio + 1)

        # Dibujar ejes cartesianos
        ax.axhline(0, color='black', linewidth=1)
        ax.axvline(0, color='black', linewidth=1)

        # Dibujar círculo
        circle = patches.Circle((0, 0), self.radio, color=self.color)
        ax.add_patch(circle)

        # Dibujar puntos (ojos)
        ax.scatter(self.puntos[:, 0], self.puntos[:, 1], color='red', s=100)

        plt.gca().set_aspect('equal', adjustable='datalim')
        plt.show()

    def escalar(self, factor):
        """
        Escala el círculo y los puntos por un factor dado.
        """
        self.radio *= factor
        self.puntos *= factor

    def desplazar(self, dx, dy):
        """
        Desplaza los puntos y el centro del círculo.
        """
        self.puntos += np.array([dx, dy])

    def producto_interno(self, punto):
        """
        Calcula el producto interno entre los puntos y otro punto dado.
        """
        return np.dot(self.puntos, punto)

    def calcular_norma(self):
        """
        Calcula la norma de los puntos respecto al origen.
        """
        return np.linalg.norm(self.puntos, axis=1)

    def rotar(self, angulo):
        """
        Rota los puntos un cierto ángulo en grados alrededor del origen.
        """
        theta = np.radians(angulo)
        rot_matrix = np.array([[np.cos(theta), -np.sin(theta)],
                               [np.sin(theta), np.cos(theta)]])
        self.puntos = self.puntos @ rot_matrix.T

# Ejemplo de uso
escena = Escena()
print("Dibujando escena inicial...")
escena.dibujar_escena()

print("Escalando la escena por un factor de 1.2...")
escena.escalar(1.2)
escena.dibujar_escena()

print("Desplazando la escena en (2, -1)...")
escena.desplazar(2, -1)
escena.dibujar_escena()

print("Rotando la escena 30 grados...")
escena.rotar(30)
escena.dibujar_escena()
\end{lstlisting}

\section{Conclusión}
Este código permite manipular una escena en 2D mediante transformaciones geométricas básicas, lo que facilita su visualización y análisis.

\end{document}