\documentclass[a4paper, 10pt]{article}
\hyphenpenalty=8000
\textwidth=125mm
\textheight=185mm

\usepackage{graphicx}
\usepackage{alltt}
\usepackage{amsmath}
\usepackage[hidelinks, pdftex]{hyperref}

\begin{document}

\begin{center}
\LARGE
\textbf{Transformaciones Geométricas en una Escena 2D}\footnote{Este trabajo fue desarrollado como parte de un proyecto de aprendizaje.}\\[6pt]
\small
\textbf{Robbyel Elias, Carlos Jerónimo}\\[6pt]
INFOTEP\\[6pt]
\end{center}

\begin{abstract}
Este documento presenta la implementación de una clase en Python denominada Escena, la cual permite realizar transformaciones geométricas sobre un círculo y un par de puntos representando ojos. Se detallan los métodos para escalar, trasladar y rotar estos elementos, así como su representación gráfica mediante la biblioteca Matplotlib. La implementación facilita la exploración de transformaciones geométricas de manera interactiva.

\textbf{Palabras clave:} Transformaciones geométricas, rotación, escalado, traslación, Python, Matplotlib.
\end{abstract}

\section{Introducción}
El estudio de las transformaciones geométricas es fundamental en diversas aplicaciones, desde el diseño gráfico hasta la simulación física. En este documento, presentamos una implementación en Python que permite visualizar y manipular elementos geométricos básicos, facilitando la comprensión de estas transformaciones.

\section{Descripción del Código}
La clase Escena implementa un círculo y dos puntos en un espacio 2D. Se han desarrollado métodos para modificar estos elementos mediante transformaciones geométricas básicas.

\subsection{Definición de la Clase}
La clase Escena se define con un radio inicial y un color para el círculo, además de la posición de dos puntos que representan ojos.

\begin{alltt}
class Escena:
    def _init_(self, radio=5, color='darkblue'):
        self.radio = radio
        self.color = color
        self.puntos = np.array([[-1.5, 2], [1.5, 2]])
\end{alltt}

\subsection{Método para Dibujar la Escena}
Este método utiliza Matplotlib para graficar el círculo y los puntos.

\begin{alltt}
    def dibujar_escena(self):
        fig, ax = plt.subplots()
        ax.set_xlim(-self.radio - 3, self.radio + 3)
        ax.set_ylim(-self.radio - 3, self.radio + 3)
        circle = patches.Circle((0, 0), self.radio, color=self.color, fill=False)
        ax.add_patch(circle)
        ax.scatter(self.puntos[:, 0], self.puntos[:, 1], color='red', s=100)
        plt.grid(True)
        plt.show()
\end{alltt}

\subsection{Transformaciones Geométricas}
La clase permite modificar la escena mediante tres tipos de transformaciones:

\begin{itemize}
    \item \textbf{Escalado}: Modifica el tamaño del círculo y la posición de los puntos proporcionalmente.
    \item \textbf{Traslación}: Mueve los puntos en una dirección determinada.
    \item \textbf{Rotación}: Gira los puntos alrededor del origen.
\end{itemize}

\begin{alltt}
    def escalar(self, factor):
        self.radio *= factor
        self.puntos *= factor

    def desplazar(self, dx, dy):
        self.puntos += np.array([dx, dy])

    def rotar(self, angulo):
        theta = np.radians(angulo)
        rot_matrix = np.array([[np.cos(theta), -np.sin(theta)],
                               [np.sin(theta), np.cos(theta)]])
        self.puntos = self.puntos @ rot_matrix.T
\end{alltt}

\section{Resultados y Conclusión}
El código proporciona una representación visual clara de las transformaciones geométricas aplicadas a la escena. A través de su ejecución, es posible observar los efectos del escalado, la traslación y la rotación en tiempo real. Este tipo de implementaciones resulta útil en aplicaciones de gráficos computacionales, videojuegos y modelado geométrico.

\end{document}