\documentclass[a4paper,10pt]{article}
\hyphenpenalty=8000
\textwidth=125mm
\textheight=185mm

\usepackage{graphicx}    % Para insertar imágenes
\usepackage{amsmath}     % Para expresiones matemáticas
\usepackage{listings}    % Para incluir código fuente
\usepackage{color}       % Para colorear el código
\usepackage[hidelinks, pdftex]{hyperref}  % Para hipervínculos

% Configuración de listings para Python
\definecolor{mygreen}{rgb}{0,0.6,0}
\definecolor{mygray}{rgb}{0.5,0.5,0.5}
\definecolor{mymauve}{rgb}{0.58,0,0.82}
\lstset{%
  language=Python,
  basicstyle=\footnotesize\ttfamily,
  numbers=left,
  numberstyle=\tiny\color{mygray},
  stepnumber=1,
  numbersep=5pt,
  backgroundcolor=\color{white},
  showspaces=false,
  showstringspaces=false,
  showtabs=false,
  frame=single,
  rulecolor=\color{black},
  tabsize=4,
  captionpos=b,
  breaklines=true,
  breakatwhitespace=false,
  keywordstyle=\color{blue},
  commentstyle=\color{mygreen},
  stringstyle=\color{mymauve},
  escapeinside={\%*}{*)}
}

\begin{document}

\begin{center}
  \Large\textbf{Análisis y Aplicación de Rotaciones y Traslaciones en Espacio 3D}\\[12pt]
  \normalsize
  Alan Miranda, Marlon Caviedes, Cristian Orozco \\
  Infotep \\
  \today
\end{center}

\begin{abstract}
En este documento se exploran las transformaciones en el espacio 3D mediante rotaciones y traslaciones. En particular, se evalúa el efecto del orden de las rotaciones y se implementa una función para traslaciones. Se incluyen fragmentos de código, ejemplos numéricos, visualizaciones y un análisis detallado de los resultados obtenidos.  
\end{abstract}

\section{Introducción}
Las transformaciones geométricas, como las rotaciones y traslaciones, son herramientas esenciales en diversas áreas (por ejemplo, gráficos por computadora, robótica y procesamiento de imágenes). En el espacio tridimensional, las rotaciones no son conmutativas, lo que significa que el orden en que se aplican puede modificar significativamente el resultado final. La presente actividad se centra en evaluar este efecto y en la implementación de una función para realizar traslaciones.

\section{Objetivos de la Actividad}
\begin{itemize}
    \item Evaluar el efecto del orden de las rotaciones en el espacio 3D.
    \item Verificar, mediante ejemplos numéricos, que las rotaciones en 3D no son conmutativas.
    \item Implementar una función para realizar traslaciones en el espacio 3D y analizar su correcto funcionamiento.
    \item Visualizar los resultados obtenidos (rotaciones y traslaciones) y extraer conclusiones sobre su comportamiento.
\end{itemize}

\section{Descripción de la Actividad}
La actividad se divide en dos partes principales:

\subsection{Parte 1: Rotaciones}
Se implementaron funciones en Python para generar las matrices de rotación alrededor de cada eje:
\begin{lstlisting}[caption={Funciones para crear matrices de rotación}, label=lst:rotaciones]
def rotation_x(theta):
    """Matriz de rotación alrededor del eje X"""
    return np.array([
        [1, 0, 0],
        [0, np.cos(theta), -np.sin(theta)],
        [0, np.sin(theta), np.cos(theta)]
    ])

def rotation_y(theta):
    """Matriz de rotación alrededor del eje Y"""
    return np.array([
        [np.cos(theta), 0, np.sin(theta)],
        [0, 1, 0],
        [-np.sin(theta), 0, np.cos(theta)]
    ])

def rotation_z(theta):
    """Matriz de rotación alrededor del eje Z"""
    return np.array([
        [np.cos(theta), -np.sin(theta), 0],
        [np.sin(theta), np.cos(theta), 0],
        [0, 0, 1]
    ])
\end{lstlisting}

A partir de un punto inicial $P_0 = [1,1,0]$, se aplicaron distintas secuencias de rotaciones (usando un ángulo de $\pi/2$, es decir, 90°) para evaluar el efecto del orden de las transformaciones. Se definieron las siguientes secuencias:
\begin{itemize}
    \item Secuencia 1: $X \rightarrow Y \rightarrow Z$
    \item Secuencia 2: $Z \rightarrow Y \rightarrow X$
    \item Secuencia 3: $Y \rightarrow X \rightarrow Z$
    \item Secuencia 4: $Z \rightarrow X \rightarrow Y$
    \item Secuencia 5: $X \rightarrow Z \rightarrow Y$
    \item Secuencia 6: $Y \rightarrow Z \rightarrow X$
\end{itemize}

Además, se probó una secuencia con ángulos diferentes (45°, 60° y 30°) para observar el efecto en la transformación.

\subsection{Parte 2: Traslaciones}
Se implementó una función para trasladar un punto en el espacio 3D. La traslación se realiza sumando al punto un vector de desplazamiento. El código utilizado es el siguiente:

\begin{lstlisting}[caption={Función para traslación en 3D}, label=lst:traslacion]
def trasladar_punto(punto, vector_traslacion):
    """
    Traslada un punto en un espacio tridimensional.
    
    Parámetros:
        punto: Vector NumPy de tres elementos (x, y, z) del punto original.
        vector_traslacion: Vector NumPy de tres elementos (dx, dy, dz).
    
    Retorna:
        El punto trasladado.
    """
    return punto + vector_traslacion
\end{lstlisting}

Se aplicó la función a un punto de prueba y se visualizó el resultado.

\section{Conclusiones}
\begin{itemize}
    \item \textbf{Rotaciones:} La actividad demuestra que el orden de las rotaciones en 3D es crucial, ya que el resultado final depende fuertemente de la secuencia de operaciones. Este comportamiento se debe a la no conmutatividad de las rotaciones en el espacio tridimensional.
    \item \textbf{Traslaciones:} La implementación de la traslación confirma que se trata de una operación sencilla de aplicar, donde el punto se desplaza de acuerdo con el vector de traslación.
    \item En conjunto, estas transformaciones (rotaciones y traslaciones) son herramientas fundamentales en la manipulación de objetos en espacios 3D y su correcta implementación es esencial para aplicaciones en diversas áreas tecnológicas.
\end{itemize}

\end{document}
